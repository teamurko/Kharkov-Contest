\begin{problem}{Разрезание апельсина}
{cutting.in}{cutting.out}
{2 seconds}{256 Mebibytes}

%Author: Stanislav Pak

Петя нашел в холодильнике апельсин. Кроме него, есть его друзья,
которые тоже не прочь полакомиться фруктом. Поэтому Петя решил
поделить апельсин на несколько частей. Для этого он ножом делает 
разрезы, проходящие через центр апельсина (можно считать апельсин 
идеальным шаром). Естественно, апельсин можно вращать на любой угол
вокруг любой оси вращения. С разрезанием Петя справился превосходно.
Теперь ему предстоит посчитать долю каждого куска, чтобы понять, 
сможет ли он справедливо распределить апельсин между друзьями.

\InputFile

Во входном файле в первой строке даны два числа $R$ и $N$ 
$(1\le R \le 100, 1 \le N \le 100)$,
радиус апельсина и количество разрезов соответственно. Каждая из 
следующих $N$ строк содержит по три числа $x_i, y_i, z_i$ ~---
координаты вектора, ортогонального плоскости, в которой лежит разрез.
Центр апельсина лежит в начале координат. Координаты по модулю не превосходят
$100$.

\OutputFile

Выведите в первую строку число $M$ ~--- количество кусков.
Во вторую строку выведите доли кусков в неубывающем порядке,
разделяя числа одним пробелом. Выводите числа с не менее чем $6$ знаками после запятой.

\Example

\begin{example}
\exmp{
1 1
1 0 0
}{
2
0.5 0.5
}%
\end{example}

\end{problem}
