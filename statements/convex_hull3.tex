\begin{problem}{Выпуклая оболочка 3D}
{convex.in}{convex.out}
{2 seconds}{256 Mebibytes}

%Author: Stanislav Pak

Даны $n$ точек в пространстве. Никакие 4 точки не лежат в одной плоскости.
Найдите выпуклую оболочку этих точек.

\InputFile

Первая строка содержит число $n$ $(4 \le n \le 10000)$. Далее, в $n$ строках даны по три числа 
~--- координаты точек. Все координаты целые, не превосходят по модулю 500.

\OutputFile

В первую строку выведите количество граней $m$. Далее в последующие $m$ строк выведите
описание граней: количество вершин и номера точек в исходном множестве. 
Точки нумеруются в том порядке, в котором они даны во входном файле. Точки в 
пределах грани должны быть отсортированы в порядке против часовой стрелки относительно
внешней нормали к грани.  

\Example

\begin{example}
\exmp{
1 1
1 0 0
}{
2
0.5 0.5
}%
\end{example}

\end{problem}
