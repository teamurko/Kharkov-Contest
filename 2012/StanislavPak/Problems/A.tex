\begin{problem}{Задача A}{1 с}{256 МБ}

Даны два выпуклых многоугольника с числом вершин N и M, соответственно,
а также два вектора, задающие их движение. Длина вектора определяет скорость
в единицах длины в секунду.
Ваша задача - выяснить, столкнутся ли эти многоугольники.

\Limit

$3 \le N, M \le 40000$
Все координаты не превосходят по модулю $10^8$

\InputFile
В первой строке содержится число вершин в первом многоугольнике $N$.
В следующих $N$ строках содержатся координаты вершин. В следующей строке
находятся координаты вектора скорости. Далее идет аналогичная информация
о втором многоугольнике.

\OutputFile
Если ответ отрицательный, выведите <<No collision>>, в противном случае
выведите время столкновения (касание считается столкновением) как несократимую
дробь в виде $x/y$. Движение начинается в момент времени $0$.

\Example
\begin{example}
\end{example}
\end{problem}
