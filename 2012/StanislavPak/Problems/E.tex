\begin{problem}{Задача E}{1 с}{256 МБ}
Агент Вася 00* Пупкин  обнаружил в своем почтовом ящике треугольное
письмо. Послание из Центра. Для соблюдения секретности все послания из Центра
снабжаются механизмом самоуничтожения, и Вася знает, что как только он вскроет
конверт, специальное вещество, нанесенное на все три вершины треугольного
послания, прореагирует с воздухом и воспламенится.\\
После того как Вася вычислил время сгорания письма и выпил чашку кофе,
перед ним встала следующая задача. От письма на столе остался
пепел треугольной формы, и теперь он хочет собрать его в одну точку
с помощью совка с прямолинейным основанием. Вася может провести
основанием совка по столу, передвигая таким образом пепел.
Вектор движения совка при этом не меняется и не обязательно ортогонален
основанию. Можно считать, что те точки треугольника, которые попали на
основание совка, не меняют своего положения относительно совка, но остаются
на столе. Так же можно считать, что ширины основания совка достаточно,
чтобы пепел заметался прямой, проходящей через основание.\\
Зафиксируем какую-нибудь точку на основании совка, тогда можно
определить расстояние, которое пройдет эта точка, пока Вася собирает пепел.
Назовем эту величину штрафом уборки.
Так как Вася все пытается оптимизировать, то сейчас его интересует, каков
минимальный штраф уборки, если он может изменять направление
движения совка не более одного раза.

\Limit

Все координаты не превосходят по модулю $10^5$.
Все числа целые.
Три вершины не лежат на одной прямой.

\InputFile
Входные данные содержат три строки. На каждой строке содержится пара чисел
~--- координаты вершины треугольника.

\OutputFile
В единственную строку выходных данных вывести одно вещественное число ~---
минимальный штраф. Ответ необходимо
выводить с не менее чем шестью знаками после десятичной точки

\Example
\begin{example}
\end{example}
\end{problem}
