\SolutionHead{Задача C}
\indent\indent\ Можно заметить и показать, что накрывающий
прямоугольник минимальной площади на одной из сторон
имеет две точки множества. Таким образом, после построения выпуклой оболочки
достаточно перебрать ее сторону $sa-sb$
($sa$, $sb$ ~--- вершины) и достроить остальные стороны
прямоугольника. Ориентируем стороны, например,
против часовой стрелки. Противоположная сторона будет проходить
через наиболее удаленную от $sa-sb$ точку $p$. Если сторона будет меняться
последовательно против часовой стрелки, то противоположная точка $p$ тоже
будет меняться последовательно против часовой стрелки.\\
Чтобы найти боковые стороны прямоугольника нужно найти стороны
выпуклой оболочки, образующие с вектором $n$ наименьший угол,
где $n$ ~--- вектор, ортогональный $sa-sb$.\\

Как можно сделать все в целых числах? Пусть две боковые
противоположные вершины многоугольника это $a$ и $b$ такие,
что вершины $a$, $p$, $b$ расположены в порядке против часовой
стрелки. Тогда длина накрывающего прямоугольника ~---
это $(sa-sb \times sa-b) / dist(sa, sb)$,
ширина ~--- это $((n \times sa-b) - (n \times sa-a)) / dist(sa, sb)$,
где $dist(a, b)$ ~--- длина вектора $a-b$. Видно, что
площадь прямоугольника ~--- представляется как отношение
целых чисел. Произведение числителей не помещается в 64-битный
целый тип, поэтому нужно сократить знаменатель с обоими числителями
и умножить два длинных числа.
