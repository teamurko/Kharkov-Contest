\SolutionHead{Задача C}
\indent\indent\ Согласно теореме 1 из лекции,
после построения выпуклой оболочки
достаточно перебрать ее сторону $\overline{sa,sb}$
($sa$, $sb$ ~--- вершины) и достроить остальные стороны
прямоугольника. Ориентируем стороны, например,
против часовой стрелки. Противоположная сторона будет проходить
через наиболее удаленную от $\overline{sa,sb}$ точку $p$.
Если сторона будет меняться
последовательно против часовой стрелки, то противоположная точка $p$ тоже
будет меняться последовательно против часовой стрелки.\\
Чтобы найти боковые стороны прямоугольника нужно найти стороны
выпуклой оболочки, образующие с вектором $\overline{n}$ наименьший угол,
где $\overline{n}$ ~--- вектор, ортогональный $\overline{sa,sb}$.\\

Как можно сделать все в целых числах? Пусть две боковые
противоположные вершины многоугольника это $a$ и $b$ такие,
что вершины $a$, $p$, $b$ расположены в порядке против часовой
стрелки. Тогда длина накрывающего прямоугольника ~---
это $(\overline{sa,sb} \times \overline{sa,b}) / dist(sa, sb)$,
ширина ~--- это
$((\overline{n} \times \overline{sa,b})
- (\overline{n} \times \overline{sa,a})) / dist(sa, sb)$,
где $dist(a, b)$ ~--- длина вектора $\overline{a,b}$. Видно, что
площадь прямоугольника ~--- представляется как отношение
целых чисел. Произведение числителей не помещается в 64-битный
целый тип, поэтому нужно сократить знаменатель с обоими числителями
и умножить два длинных числа.
