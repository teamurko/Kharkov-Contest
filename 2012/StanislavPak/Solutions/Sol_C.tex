\SolutionHead{Задача C}
\indent\indent\ Можно заметить и показать, что накрывающий
прямоугольник минимальной площади на одной из сторон
имеет две точки множества. Таким образом, после построения выпуклой оболочки
достаточно перебрать ее сторону $s$ и достроить остальные стороны
прямоугольника. Ориентируем стороны, например,
против часовой стрелки. Противоположная сторона будет проходить
через наиболее удаленную от $s$ точку $p$. Если сторона будет меняться
последовательно против часовой стрелки, то противоположная точка $p$ тоже
будет меняться последовательно против часовой стрелки.\\
Чтобы найти боковые стороны прямоугольника нужно найти стороны
выпуклой оболочки, образующие с вектором $n$ наименьший угол,
где $n$ ~--- вектор, ортогональный $s$.
