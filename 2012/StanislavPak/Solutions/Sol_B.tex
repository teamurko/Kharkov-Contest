\SolutionHead{Задача B}
\indent\indent\
Диаметры множества точек и их выпуклой оболочки совпадают,
потому что любые две внутренние точки не дают максимального
расстояния. Точки, дающие максимум не могут лежать строго на стороне,
они обязательно совпадают с вершинами выпуклой оболочки
по теореме косинусов, например. Тем не менее число вершин
в выпуклой оболочке может быть большим, перебирать
пары точек нельзя.\\
Рассмотрим пару различных вершин $a$ и $b$ выпуклой оболочки $P$. Проведем
через них прямые, ортогонально отрезку $ab$, они задают бесконечную
полосу. Если все точки $P$ лежат
между этими прямыми (или на них, то есть на полосе),
то назовем эту пару вершин противоположными.
Ясно, что максимум расстояния достигается на противоположных точках,
потому что в противном случае ответ можно улучшить, опять же
по теореме косинусов.
Оказывается, противоположных точек линейное от размера $P$ число,
их можно все перебрать.\\
Зафиксируем порядок вершин $P$, например, против часовой стрелки. Зафиксируем
вершину $p$. Пусть предыдущая $pp$, следующая $pn$. Найдем первую
по порядку после вершины $p$ вершину $ps$ такую, что она максимально
удалена от прямой $pp-p$. В силу монотонности следующая после $ps$ вершина
находится не дальше. Также найдем последнюю после $p$ вершину $pe$,
которая максимально удалена от прямой $p-pn$. Все вершины, которые образуют
противоположную пару с $p$ лежат между $ps$ и $pe$ включительно,
потому другие вершины вместе в $p$ порождают полосу, не содержащую
либо $pp$, либо $pn$. При переходе от рассмотрения вершины $p$ к рассмотрению
вершины $pn$, аналогичные вершины $ps$ и $pe$ могут только сдвинуться
только дальше по часовой стрелке. Понятно, что если в выпуклой оболочке нет
трех подряд идущих коллинеарных точек (лежащих на одной прямой), то
следующая вершина $ps$ может продолжаться либо от $pe$, либо от вершины,
непосредственно предшествующей $pe$. Таким образом, суммарно
будет рассмотрено не более $O(size P)$ пар вершин.\\

Комментарий. Можно все делать в целых числах, пользуясь векторным
произведением, чтобы найти треугольник с наибольшей высотой.
