\SolutionHead{Задача B}
\indent\indent\
В данную задачу можно решать методом указателей, бегущих
по вершинам выпуклого многоугольника, как рассказано в лекции.
Чтобы сравнить вершины $c$ и $d$ по расстоянию до прямой,
проходящей через сторону $\overline{a,b}$, достаточно
сравнить их векторные произведения $\overline{a,b}\times\overline{a,c}$
и $\overline{a,b}\times\overline{a,d}$, потому что они выражают
площадь параллелограмма, натянутого на вектора, и, с другой стороны,
площадь равна произведению высоты (расстояния) на фиксированное основание
$\overline{a,b}$.
