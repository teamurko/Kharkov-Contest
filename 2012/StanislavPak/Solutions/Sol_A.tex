\SolutionHead{Задача A}
\indent\indent\ Пусть $P$, $Q$ ~--- многоугольники, которые движутся
со скоростями $vp$ и $vq$.
Зафиксируем некоторый момент $t \ge 0$, как узнать, что многоугольники
$P(t) := P + vp \cdot t$ и $Q(t) := Q + vq \cdot t$ пересекаются?
Отразим второй многоугольник относительно начала координат в $Q'(t)$,
и найдем прямую сумму (или сумму Минковского) $P(t)$ и $Q'(t)$. Многоугольники
$P(t)$ и $Q(t)$ имеют общую точку тогда и только тогда, когда их сумма
содержит точку $(0, 0)$. Можно легко показать, что сумма Минковского для
выпуклых множеств выпукла и сдвигается на вектор, если аргумент суммы
сдвигается на вектор. Таким образом, сумма $P(t)$ и $Q'(t)$ есть сумма
$P$ и $Q'$ (центрально симметрично отраженный относительно начало координат),
сдвинутая на вектор $vp - vq$, сумма движется вдоль вектора $vp - vq$, или
можно думать, что точка $(0, 0)$ движется вдоль вектора $vq - vp$. Нужно найти
момент, когда луч из начала координат, распространяющийся со скоростью
$vq - vp$ пересечет сумму $P$ и $Q'$.

%\indent\indent\ - îñòàâèòü êàê åñòü
