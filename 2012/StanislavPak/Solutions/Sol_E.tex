\SolutionHead{Задача E}
\indent\indent\ Ясно, что всегда придется изменить направление,
потому что треугольник невырожден. Наблюдением можно установить,
что в задаче нужно найти накрывающий параллелограмм минимального
периметра. Ответом будет его полупериметр.\\
Согласно теореме 1 из лекции, граница оптимального параллелограмма
содержит сторону треугольника. При фиксированной стороне $ab$
оптимальный параллелограмм есть либо прямоугольник, если на стороне
$ab$ нет тупых углов, либо параллелограмм, вторая сторона которого
равна второй стороне тупого угла.

